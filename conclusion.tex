\chapter{Conclusions and Final Remarks}
\label{ch:conclusions}

The focus of this thesis is to study users' reactions and evolution on social networks. In the second chapter, our efforts are directed at the problem of missed reactions. We propose a text similarity method to identify missed reactions and validate it on Twitter data. We estimate the potential missed behavior from users that are not captured by current Twitter mechanisms and we identified types of users that are significantly underrepresented based solely on these metrics. In the third chapter, we analyze Reddit users' evolution from a cohort perspective built on top of the time that they joined the network. We found wide differences in behavior depending on the year that they joined the network as well as how not account for temporal effects can be misleading. This thesis gives us better understanding of how we might ignore or misrepresent users' behavior on social networks and provide methods and analysis tools that help us to avoid such problems in the future. Here summarize our findings and some of their implications.

\section{Reaction Detection}

The first part of this thesis presented a novel method of capturing some of a user's non-explicit reactions to followees' content in Twitter by using text similarity scores between a user's tweets and those of their followees \cite{Barbosa}.  The analysis indicates that the method does generate higher scores on average for system tagged Replies and Retweets than Non-Tagged tweets, suggesting that it captures real signal about responses (\ref{rq:similarityPotential}).  Using a conservative cutoff for predicting whether a non-tagged tweet is a response suggests that at least \highNonTaggedTweetCountPct{}\% of actual responses are not tagged by the system.  These responses are distributed across almost a quarter of the users in the dataset, with a quarter of those having more missed reaction messages than explicit system tagged ones. These are not just naive, low-activity users who do not understand Twitter and might be ignored in analysis; a number of these users are quite active, with dozens or hundreds of tweets in a 14-day window (\ref{rq:usersDistribution}).  

Although the method has provided useful insights into the prevalence of non-explicit replies in Twitter, it is a coarse model.  It tends to under-evaluate Replies; is more sensitive to Retweet size than desirable; likely misses a number of non-explicit responses that have lower scores but are nonetheless real responses to the feed; and doesn't address responses to content outside the feed such as views by hashtag or username.  Ongoing work aims at addressing these limitations by improving the quality of the scoring function.  One natural way of improving the scoring function is to incorporate other relevant social features highlighted by past work (Table~\ref{tab:characteristics}).  We expect that better models of language, network characteristics, and attention that build on these features would give better estimates of how people react to content produced by their followees.

Another possible unfolding research topic is how to use these reaction scores to understand the reaction patterns and estimate the individual reaction level for each user.  This is important for effective models of diffusion at all levels, from understanding when adding an individual to a follower network might be most valuable, to estimating the overall reach of an individual's network, to modeling diffusion of information in the large.  Missing \highNonTaggedTweetCountPct{}\% of responses and \usersAboveLinePct{}\% users is a substantial amount of error to bear for such models, making the identification of non-explicit responses an important problem to pursue.


\section{Users' Behavior Evolution}

The second part of this thesis highlights the importance of taking time into consideration when analyzing users' evolution in social networks. We do so by cohorting the users based on their creation year \cite{Barbosa2016}. Although simple, this approach provides a number of insights that would be missed by straightforward aggregate analysis methods.  We also analyze the evolution of users and communities from a shifted time referential: considering the time of an action in relation to the user creation date. This also reveals unexpected phenomena that we would otherwise not notice.

While analyzing how the amount of posting changes over time (\ref{rq:activityChange}), we found that user posting activity for surviving Reddit users is actually significantly higher than a naive average would suggest, that older users who survive are considerably more active than younger survivors, and that these newer users are unlikely to catch up (\ref{srq:newFollowOld}).  Controlling for survival provided evidence for hypothesis (\ref{hyp:lowActivityLeave}), that users have a stable level of posting activity over time (with slightly decreasing patterns).  Further, the percentage of surviving but low-activity users is increasing in the younger cohorts 

When looking at changes in comment length over time (\ref{rq:commentLengthOverTime}) as a proxy for users' effort, we found that while the overall average in Reddit seems to decrease, users actually write longer comments as they survive, no matter when they join.  However, later cohorts of users that joined the network are writing smaller comments; their greater number leads to an instance of Simpson's paradox, where the overall average decreases while the series for each individual cohort increases. 

Finally, we analyzed whether users change their commenting versus submission behavior over time (\ref{rq:typeOfActivity}). 
We found that users with a higher initial comment to submission ratio survive longer on average, and that this ratio increases for surviving users, particularly for earlier cohorts.  This isn't because activity rises overall, as posting activity remains stable; instead, it suggests that longer-term users substitute commenting for submissions. 

An important remark of this thesis is how different demographics of users joining and leaving a network play a significant role in shaping the average user behavior. Failing to account for these might limit our interpretation of the data (\ref{hyp:survivingMorePosts}, \ref{hyp:decreasingCommitment} or \ref{hyp:communityNorm}) and lead to wrong conclusions.

Both our work and its limitations suggest fruitful directions for better understanding of users' evolution in both Reddit and online communities in general, directions we hope inspire other works in this area.  
