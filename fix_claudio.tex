### cap 2:

% Sam: expliquei melhor como funciona a similaridade que calculamos, coloquei uma
% tabela de exemplo
- uso de tf-idf, tabela com ex de tweetes iguais, similaridade
- criar uma se��o apresentando a quest�o levantada pelo Claudio e respondendo-a. Sua resposta: � menos importante perder alguns casos, do que garantir que o que foi pego est� correto.
- explicar melhor o caso de tweets

% Sam: Secao com indicacoes de pessoas que falam para copiar e colar e da intuicao sobre quem sao esses 8% de pessoas
- refer�ncia: n�o retweet, poste novamente a mesma coisa
- intui��o sobre quem s�o os 8% que n�o retweetam. Por que eles trabalham mais?

% Sam: Secao que fala de limitacoes, contexto e referencia conteudos alem de texto
- retweet de imagem, refer�ncia do Paulo da IBM
- contexto compartilhado: enfatizar
- passos a mais: encontrar pessoas mais influentes, retirar o contexto compartilhado dos dados (eg do gol em jogos de futebol, debates eleitorais da TV, etc)

3 possibilidades:
a) usar um m�todo alternativo e mostrar que os resultados se mant�m
b) dado simulado e mostrar que ela comporta de maneira esperada
c) provar que a m�trica se comporta como esperado
% Sam: minha opcao que nao existia na lista: d) mostrar resultados semelhantes de outros pesquisadores


### cap 3:

- quais as mensagens do cap?
- tabela com research questions 

% Sam: Adicionei graficos sem escala em log
- X gr�fico do tamanho da rede em log n�o enfatizou o efeito. colocar o gr�fico sem log tb
- faltou discutir o tamanho da rede ao longo do tempo
- usu�rios entram e saem
% Nao da pra saber quantos saem e quantos entram, so o net value disso
% para isso vc precisa de uma definicao de ''ativo``, que eh o q a gente faz
% Mesmo nessa definicao, identifica quantos sairam e quantos entraram nao eh trivial - preciso codar isso e nao esta facil
- dados de entrada e sa�da, quantos entram e saem 


- quest�o em que cometeu equ�vocos, como n�mero de likes, upvotes, downvotes, 
- ao usar #posts e tamanho, a an�lise n�o est� cometendo um equ�voco?
- incluir outros resultados de an�lises que foram feitas com outros dados e outras an�lises, que n�o entraram no paper.

mandar uma nova vers�o pro Claudio e para mim indicando claramente as mudan�as da nova vers�o


