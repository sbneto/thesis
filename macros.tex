\newcounter{rqcounter}[chapter]
\setcounter{rqcounter}{0}
\newcounter{subrqcounter}[rqcounter]
\setcounter{subrqcounter}{0}
\newcommand{\ResearchQuestion}{\refstepcounter{rqcounter}\textbf{Ch.\thechapter~RQ\arabic{rqcounter}}}
\renewcommand{\therqcounter}{\textbf{Ch.\thechapter~RQ\arabic{rqcounter}}}
\newcommand{\SubResearchQuestion}{\refstepcounter{subrqcounter}\textbf{Ch.\thechapter~RQ\arabic{rqcounter}\alph{subrqcounter}}}
\renewcommand{\thesubrqcounter}{\textbf{Ch.\thechapter~RQ\arabic{rqcounter}\alph{subrqcounter}}}

\newcounter{hypcounter}
\setcounter{hypcounter}{0}
\newcommand{\Hypothesis}{\refstepcounter{hypcounter}\textbf{Ch.\thechapter~H\arabic{hypcounter}}}
\renewcommand{\thehypcounter}{\textbf{Ch.\thechapter~H\arabic{hypcounter}}}

\newcommand{\thesisPages}{60}
\newcommand{\thesisDay}{1st}
\newcommand{\thesisMonth}{July}
\newcommand{\thesisYear}{2016}
\newcommand{\surnameAbbr}{Barbosa Neto, S. M.}
\newcommand{\phdTitle}{Revealing social networks' missed behavior: detecting reactions and time-aware analyses}
\newcommand{\phdTitleBr}{Revelando o comportamento perdido dos usuários: detectando reações e análises temporais}

%esciece15
\newcommand{\joinNameTweet}[2] {
\texttt{\textbf{#1:} #2}
}

\newcommand{\tweetRowFull}[4] {
\stepcounter{rowcount}
\multirow{2}{0.3cm}{ \therowcount }	& 
\multirow{2}{0.7cm}{ $#2$ } 	& 
#3 \\
&& 
\cellcolor{gray!25} #4 \\ \cline{1-3}
}

\newcounter{rowcount}
\newcommand{\tweetTableFull}[3] {
%\onecolumn
\setcounter{rowcount}{0}
\begin{table*}[!htbp]
	\centering
	%\footnotesize
	\fontsize{8pt}{9pt}\selectfont
		%\hspace*{-0.5cm}  
		\setlength\tabcolsep{0.1cm}
		\begin{tabularx}{\linewidth}{|>{\raggedright\arraybackslash}m{0.3cm}|>{\raggedright\arraybackslash}m{0.8cm}|>{\raggedright\arraybackslash}m{14cm}|}
			\cline{1-3}
			\centering\arraybackslash \textbf{\#} & \centering\arraybackslash \textbf{Score} & \centering\arraybackslash \textbf{Retweets} \\ \cline{1-3} 
			\input{#3}
			%\multicolumn{3}{l}{}\\ \cline{1-3}
			%\centering\arraybackslash \textbf{\#} & \centering\arraybackslash \textbf{Score} & \centering\arraybackslash \textbf{Replies} \\ \cline{1-3} 
			%\input{#4}
			%\multicolumn{3}{l}{}\\ \cline{1-3}
			%\centering\arraybackslash \textbf{\#} & \centering\arraybackslash \textbf{Score} & \centering\arraybackslash \textbf{Non-Tagged} \\ \cline{1-3} 
			%\input{#5}
		\end{tabularx}
	\caption{#1}
	\label{tab:#2}
\end{table*}
}

%www16
\makeatletter
\define@key{subimagedic}{width}{\def\width{#1}}
\define@key{subimagedic}{scale}{\def\scale{#1}}
\makeatother
\NewDocumentCommand\subimage{O{} m G{}}{
    \begingroup
        \setkeys{subimagedic}{width={1.0},scale={1.0}, #1}
        \begin{subfigure}{\width\textwidth}
            \expandafter\includegraphics\expandafter[scale=\scale]{#2}
        \caption{#3}\end{subfigure}
    \endgroup
}