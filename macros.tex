\newcounter{rqcounter}[chapter]
\setcounter{rqcounter}{0}
\newcounter{subrqcounter}[rqcounter]
\setcounter{subrqcounter}{0}
\newcommand{\ResearchQuestion}{\refstepcounter{rqcounter}\textbf{Ch.\thechapter~RQ\arabic{rqcounter}}}
\renewcommand{\therqcounter}{\textbf{Ch.\thechapter~RQ\arabic{rqcounter}}}
\newcommand{\SubResearchQuestion}{\refstepcounter{subrqcounter}\textbf{Ch.\thechapter~RQ\arabic{rqcounter}\alph{subrqcounter}}}
\renewcommand{\thesubrqcounter}{\textbf{Ch.\thechapter~RQ\arabic{rqcounter}\alph{subrqcounter}}}

\newcounter{hypcounter}
\setcounter{hypcounter}{0}
\newcommand{\Hypothesis}{\refstepcounter{hypcounter}\textbf{Ch.\thechapter~H\arabic{hypcounter}}}
\renewcommand{\thehypcounter}{\textbf{Ch.\thechapter~H\arabic{hypcounter}}}

\newcommand{\thesisPages}{68}
\newcommand{\thesisDay}{29th}
\newcommand{\thesisMonth}{May}
\newcommand{\thesisYear}{2017}
\newcommand{\finalMonth}{July}
\newcommand{\surnameAbbr}{Barbosa Neto, S. M.}
\newcommand{\phdTitle}{Revealing social networks' missed behavior: detecting reactions and time-aware analyses}
\newcommand{\phdTitleBr}{Revelando o comportamento perdido em redes sociais: detectando reações e análises temporais}

%esciece15
\newcommand{\joinNameTweet}[2] {
\texttt{\textbf{#1:} #2}
}

\newcommand{\tweetRowFull}[4] {
\stepcounter{rowcount}
\multirow{2}{0.3cm}{ \therowcount }	& 
\multirow{2}{0.7cm}{ $#2$ } 	& 
#3 \\
&& 
\cellcolor{gray!25} #4 \\ \cline{1-3}
}

\newcounter{rowcount}
\newcommand{\tweetTableFull}[4] {
%\onecolumn
\setcounter{rowcount}{0}
\begin{table*}[!htbp]
	\centering
	%\footnotesize
	\fontsize{8pt}{9pt}\selectfont
		%\hspace*{-0.5cm}  
		\setlength\tabcolsep{0.1cm}
		\begin{tabularx}{\linewidth}{|>{\raggedright\arraybackslash}m{0.3cm}|>{\raggedright\arraybackslash}m{0.8cm}|>{\raggedright\arraybackslash}m{14cm}|}
			\cline{1-3}
			\centering\arraybackslash \textbf{\#} & \centering\arraybackslash \textbf{Score} & \centering\arraybackslash \textbf{#4} \\ \cline{1-3} 
			\input{#3}
		\end{tabularx}
	\caption{#1}
	\label{tab:#2}
\end{table*}
}

%www16
\makeatletter
\define@key{subimagedic}{width}{\def\width{#1}}
\define@key{subimagedic}{scale}{\def\scale{#1}}
\makeatother
\NewDocumentCommand\subimage{O{} m G{}}{
    \begingroup
        \setkeys{subimagedic}{width={1.0},scale={1.0}, #1}
        \begin{subfigure}{\width\textwidth}
            \expandafter\includegraphics\expandafter[scale=\scale]{#2}
        \caption{#3}\end{subfigure}
    \endgroup
}


\newcommand{\axisPic}[3] {
\begin{tikzpicture}
  \node (img1)  {\begin{varwidth}{\linewidth}#1\end{varwidth}};
  \node[below=of img1, node distance=0cm, yshift=1cm,font=\color{black}] {#2};
  \node[left=of img1, node distance=0cm, rotate=90, anchor=center,yshift=-0.7cm,font=\color{black}] {#3};
\end{tikzpicture}
}

% \begin{tikzpicture}
%   \node (img1)  {\includegraphics[scale=0.225]{example-image}};
%   \node[below=of img1, node distance=0cm, yshift=1cm,font=\color{red}] {x-axis};
%   \node[left=of img1, node distance=0cm, rotate=90, anchor=center,yshift=-0.7cm,font=\color{red}] {y-axis};
%   \node[right=of img1,yshift=0.1cm] (img2)  {\includegraphics[scale=0.25]{example-image}};
%   \node[below=of img2, node distance=0cm, yshift=1cm,font=\color{red}] {x-axis};
%   \node[left=of img2, node distance=0cm, rotate=90, anchor=center,yshift=-0.7cm,font=\color{red}] {y-axis};
% \end{tikzpicture}