\chapter{Introduction}

% Here I intent to say that we miss behavior and make mistakes when dealing with user data
% Grab supporting research from both intros and maybe look for more
% Add overview: Chapter for what we miss (escience2015) and Chapter for our mistakes (www2016)
% Decide for an overall conclusion or a ``per chapter'' conclusion, with a smaller overall conclusion

\section{Better Understanding Users' Behavior}

Users' behavior span over a multitude of actions on social networks \cite{Benevenuto2009, Gilbert2009, Comarela2012}: they search for friends, write messages, post images, videos and sounds among many other possibilities. Much of this content is captured by social networks mechanisms that explicitly tag and identify characteristics of the users' interaction. These mechanisms, however, are limited in capturing users' intention and diverse behavior, for instance, researchers have found that users might use references to content that only a specific audience from their peers can understand, turning a common context into a tool of privacy in public contexts \cite{Boyd2011}. The problem of identifying part of these interactions that are not fully captured is important to better understand users as well as to provide new perspectives on what kind of features social networks should provide.

Research has shown that not accounting for time can lead to mistakes when dealing with social networks. Just as with offline contexts, systems ans society changes over the years, as well younger generations have a different behavior from older generations. More than just considering different snapshots over time, we analyze users that joined at different stages of the network evolution. Just as with missed reactions, considering time-evolving users is important to reveal behavior that otherwise would not be noticed. Using simple cohorting methods, we demonstrate how different this behavior can be and how easy is to draw wrong conclusions from averaging practices.

In the first part of this work, we address the problem of missed reactions, proposing a text similarity method to identify missed reactions and validating it on Twitter data. We show that a considerable number of users' reactions are not captured by current mechanisms in Twitter and that some types of users are significantly underrepresented based solely on these metrics. In the second part of this work, we analyze the users' evolution from a cohort perspective built on top of the time that they joined the network. We show, in the context of Reddit, how users' behavior vary depending on the year that they joined the network as well as how misleading not accounting for time can be. 

\section{Objectives}

The objective of this work is to improve our understanding of users' behavior on social networks. More specifically, understand the missed behavior in the form of reactions that are not captured and temporal trends that are missed due to poor aggregation of data. We aim at developing methods that help us to capture and identify missed reactions and propose analysis tools that allow us to avoid misunderstanding our data.

\section{Contributions}

We can summarize the contributions of this work as the following:

\begin{itemize}
	\item Proposal of the problem of studying users' indirect reactions.
	\item Development of a method to detect indirect reactions in the context of Twitter, revealing about 11\% of missed reactions, as well as patterns of behavior that are common, but not modeled, e.g., group conversations.
	\item A cohorted view of the users' evolution on Reddit over 7 years, revealing different behavioral patterns as a function of the users' tenure in the network.
	\item Practical examples of common aggregation practices that lead to wrong conclusions when dealing with time series.
\end{itemize}


\section{Organization of this Work}

We dedicate Chapter \ref{ch:reactions} of this work to the problem of detecting missed reaction. We propose a simple method based on text similarity to detect reactions other than retweets and replies in Twitter, and we show that a significant amount of reactions are being missed. Not only that, we also show that many users consistently react in ways that are not captured by these mechanisms.

In Chapter \ref{ch:cohorts}, we cohort users in Reddit based on their creation and survival years and we show how naive aggregation of data can lead to wrong conclusions. More than that, these simple methods can reveal evolution trends that would be otherwise missed, and they highlight the significant role of users joining and leaving the network in shaping the overall behavior.

Finally, in Chapter \ref{ch:conclusions}, we discuss and summarize our findings, also providing a discussion of their impacts and possible future venues of research to pursue.