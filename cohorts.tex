\chapter{Averaging Gone Wrong: Using Time-Aware Analyses to Better Understand Behavior}
\label{cohorts}

\section{Time matters} 

\subsection{Why accounting for time is important}

Communities grow and, with time, die. For any community, its users play a role in its evolution, but they are also simultaneously affected by the evolution of the community. Untangling this interplay can help make sense of patterns of activity in a community.

One useful way to understand the evolution of a community and its users is through time, as it provides a linear account of the growth (or decay) of overall activity, types of content, and social norms and structure.  One aspect of time often considered is the tenure of a user in the community, as in studies around modeling users' preferences \cite{McAuley2013} or analyzing the evolution of their language \cite{Danescu-niculescu-mizil2013}.  These analyses uncover insights about the lifecycle of a user in a community: users' preferences and behavior change with their age in a community \cite{Panciera2010}, while their early experiences and activity shape future outcomes predictably \cite{Tan2015,Yang2009,Panciera2009, Miller2015}. 

However, much past work on online communities ignores the time at which a user joins the community and analyzes all users together.
This might be a mistake: communities may grow denser or sparser with time \cite{Leskovec2005}, develop new norms \cite{Kooti2010}, and enact policies and rules guiding people's behavior \cite{Butler2008}.
These changes mean that people experience different versions of a community at different times, which can, in turn, affect their observed behavior. This interaction with the state of a community can confound conclusions about people's behavior, because the differences one observes may simply due to changes in the community, rather than any significant change in the outcome variable of interest or the user population.  


\subsection{Cohorts are analytically useful}

A common method to control for such confounds is cohort analysis, widely used in fields such as sociology \cite{Mason2012,Glenn2005}, economics \cite{Attanasio1993,Beldona2005}, and medicine \cite{Howartz1996,Davis2010}. A cohort is defined as a group of people who share a common characteristic, generally with respect to time. For example, people born in the same year, or those who joined a school at the same time, or got exposed to an intervention at similar times can be considered as cohorts.  People in a cohort are assumed to be exposed to the same state of the world and thus are more comparable to each other than to people in other cohorts. 

For example, sociological studies often use students who join a school in the same year to understand the effect of interventions \cite{Goyette2008,Alexander2012}, and condition on the year in which people were born to understand people's  behavior, such as variations in financial decision-making \cite{Attanasio1993} or opinions on issues \cite{Firebaugh1988,Jennings1996}. Similarly, medical studies interpret effects of drugs using cohorts of people within the same age group or amount of exposure to correlated conditions \cite{Howartz1996,Davis2010}.  

Recent work shows that cohorts' importance transfers to online communities as well. Just as people's behavior varies according to their biological age, their experience in an online community may vary with their age in the community and their year of joining. In Wikipedia, we find substantial differences in the activities of cohorts of users who joined earlier versus those who joined later \cite{Welser2011}. Similarly, on review websites, users who join later tend to adopt different phrases than the older users who had joined earlier \cite{Danescu-niculescu-mizil2013}.

\subsection{What might cause these differences?}

These differences in activity between cohorts may be due to a number of reasons.  One plausible explanation is selection effects: people who are enthusiastic about a community or its goals are more likely to self-select as early members of a community, while others may be more likely to join later \cite{Li2008}.  In this case, users who join earlier might be expected to be more active, committed users than those who join later. 

Another possible explanation is that community norms may change over time.  In many cases, it is a bottom-up process. Kooti et al. showed that social conventions can define the evolution of a community and the early adopters play a major role in designing these conventions, consciously or not \cite{Kooti2010}. Examples include adoption of `RT', a retweeting norm by Twitter users and the subsequent introduction of the Retweet button on Twitter \cite{Kooti2010}; change in language use between new and old users on review websites \cite{Danescu-niculescu-mizil2013}; and assumptions of clear roles and responsibilities on Wikipedia \cite{Kittur2007a}. In other cases, it may be directed by the community managers. For instance, the makers of Digg unilaterally changed the nature of the community by introducing a new version of the website, leading to a sudden change in norms and behavior in the community \cite{Ingram2014,Lardinois2014}. 

The growth of a community may also affect people's behavior. Successful communities often grow very rapidly, which can be both good and bad for users' experience. On one hand, growth would imply availability of a larger chunk of content to choose from. On the other, it might be harder to connect to others and get responses in a bigger community. A community may also need to adopt new rules and policies to manage growth and newcomers, as in the evolution of Wikipedia \cite{Choi2010,Bryant2005}. In those cases, the experience of later cohorts of users may be vastly different from the initial ones who joined before formal rules were in place. 

Finally, patterns of use may change because the overall population of Internet users is still changing.  As more and different people come online, their influx may lead to changes in activity patterns and communities (as with the yearly entry of college freshmen, and eventually all of AOL, gaining access to Usenet).  The gradual penetration of technology also has age-related effects:  people who did not grow up in a technological environment differ in their social media and search usage compared to younger generations\cite{Correa2010,Beldona2005}. 

\subsection{Is Reddit getting ``worse'' over time?}
All of the above reasons suggest that users from different cohorts are likely to be different, which has also been demonstrated in online and offline communities \cite{Ryder1965,Danescu-niculescu-mizil2013,Prensky2001,Correa2010}.  Further, they suggest a general hypothesis that communities ``get worse'' over time because newer users are likely to be less committed and knowledgeable about the community. 

To address this hypothesis, we analyze both aggregate and cohort-based measures of user quality that are often raised about online communities: how active are users \cite{Scellato2011,Hughes2009,Java2007,Levy1984}, how much do they contribute \cite{Scellato2011,Gruhl2004,Guo2009}, and what kinds of work do they engage in \cite{Welser2011,Choi2010,Panciera2009}?  

We do this in the context of Reddit, a community that has been studied by many researchers \cite{Gilbert2013,Stoddard2015,Bergstrom2011,Tan2015}. We begin with a brief overview of both Reddit and the dataset that we use in this paper, focusing on aspects that directly impact our analyses\footnote{There is more to say about Reddit itself (see \cite{AboutReddit}).}.

\section{Data: Reddit as a community}

%% DC 14: Moving this to the end of prev work to give a little more context and beef; things felt redundant between that and this and it feels better there.

\subsection{What is Reddit, briefly}

%% Sam 11: Improving the caption
\begin{figure}[!tb]
\centering
\includegraphics[width=0.45\textwidth,natwidth=964,natheight=823]{./images/reddit.png}
\caption{Reddit interface when visualizing a submission. This is Patrick Stewart's ``AmA'' (ask me anything) in ``IAmA'' (I am a), a submission where he answers users' questions in the comments. We can see the most upvoted comment and Patrick's answer right below.}
\label{fig:reddit}
\end{figure}

Reddit is one of the largest sharing and discussion communities on the Web.  According to Alexa, as of late 2015 Reddit is in the top 15 sites in the U.S. and the top 35 in the world in terms of monthly unique visitors.  It consists of a large number of subreddits (853,000 as of June 21st, 2015\footnote{\cite{RedditStatistics} provides more statistics about Reddit.}), each of which focuses on a particular purpose.  Many subreddits are primarily about sharing web content from other sites: in ``Pics'', ``News'', ``Funny'', ``Gaming'', and many other communities, users (``Redditors'') make ``submissions'' of links posted at other sites that they think are interesting.  In other subreddits, Redditors primarily write text-based ``self-posts'': ``AskReddit'', ``IAmA'', and ``ShowerThoughts'' are places where people can ask questions and share stories of their own lives.  Generically, we will refer to submissions and text posts as ``submissions''.  

Each submission can be imagined as the root of a threaded comment tree, in which Redditors can comment on submissions or each other's comments.  Redditors can also vote on both submissions and comments; these votes affect the order in which submissions and comments are displayed and also form the basis of ``karma'', a reputation system that tracks how often people upvote a given Redditor's comments and submissions. We can observe these elements in Figure~\ref{fig:reddit}. 
%% DC 14: Not so relevant here, so cutting. Redditors can also create subreddits and volunteer to moderate them.

We choose Reddit as our target community for a number of reasons.  It has existed since 2005, meaning that there has been ample time for the community to evolve and for differences in user cohorts to appear.  Second, it is one of the most popular online communities, allowing different types of contributions---comments and original submissions---across many different subreddits.  Third, a number of Reddit users believe that it is, in fact, getting worse over time\cite{RedditWorse1,RedditWorse2,RedditWorse3,RedditWorse4,RedditWorse5,RedditWorse6}. Finally, Reddit data are publicly available through an API.

\subsection{The dataset}

\begin{figure*}[!tb]
\centering
\subimage[width=0.48, scale=0.40]{./images/cumulative_users_subreddits.eps}
\subimage[width=0.48, scale=0.40]{./images/active_users_subreddits.eps}
\caption{Figure (a) shows the cumulative growth of Reddit for users and subreddits. Figure (b) shows the number of active users and subreddits in Reddit over time. An active user or subreddit is one that had at least one post (comment or submission) in the time bin we used---here, discretized by month.}
\label{fig:cumulative}
\end{figure*}

Redditor \textit{Stuck\_In\_The\_Matrix} used Reddit's API to compile a dataset of almost every publicly available comment\cite{RedditDataset1} from October 2007 until May 2015.  The dataset is composed of 1.65 billion comments, although due to API call failures, about 350,000 comments are unavailable.  He also compiled a submissions dataset for the period of October 2007 until December 2014 (made available for us upon request) containing a total of 114 million submissions.  These datasets contain the JSON data objects returned by Reddit's API for comments and submissions\footnote{A full description of the JSON objects is available at \cite{RedditAPI}.}; for our purposes, the main items of interest were the UTC creation date, the username, the subreddit, and for comments, the comment text.

\looseness=-1
We focus on submissions and comments in the dataset because they have timestamps and can be tied to specific users and subreddits, allowing us to perform time-based analyses.   In some analyses, we look only at comments; in some, we combine comments and submissions, calling them \textbf{``posts''}.  We would also like to have looked at voting behavior as a measure of user activity\footnote{This would also give us more insight than usual into lurkers' behavior; we'll return to this in the discussion.}, but individual votes with timestamps and usernames are not available through the API, only the aggregate number of votes that posts receive.

\subsection{Preprocessing the dataset}

\looseness=-1
To analyze the data, we used Google BigQuery\cite{BigQuery}, a big data processing tool.
Redditor \textit{fhoffa} imported the comments into BigQuery and made them publicly available\cite{RedditDataset2}.  We uploaded the submission data ourselves using Google's SDK.

For the analysis in the paper, we did light preprocessing to filter out posts by deleted users, posts with no creation time, and posts by authors with bot-like names\footnote{Ending with ``\_bot'' or ``Bot''; or containing ``transcriber'' or ``automoderator''.}.
%% DC 14: Would like to at least informally be able to address the critique that this doesn't capture all bot activity and excludes some legit users by giving a feel for the scope of the problem.

We also considered only comment data from October 2007 until December 2014 in order to have a matching period for comments and submissions. After this process, we had a total of 1.17 billion comments and 114 million submissions.

\subsection{An overview of the dataset}

Here we present an overview of the dataset that shows Reddit's overall growth.  Figure~\ref{fig:cumulative}a presents the cumulative number of user accounts and subreddits created as of the last day of every month. After an initial extremely rapid expansion from 2008--2009, the number of users and subreddits have grown exponentially.  As of the end of 2014, about 16.2 million distinct users and 327 thousand subreddits made/received at least one post based on our data.

%% Sam 8: There was a problem in my query for the active users, I was counting less than I should. I replaced the text with the right numbers, but they might not be considered ``and order of magnitude less'' now, more like 5x less
However, as with many other online sites, most users \cite{Scellato2011,Hughes2009,Java2007} and communities \cite{Arguello2006} do not stay active. We define as an ``\textbf{active user}'' one that made at least one post in the month in question. Similarly, an ``\textbf{active subreddit}'' is one that received at least one post in the month. In December 2014, about 2.7 million users and 66 thousand subreddits were active, both around a fifth of the cumulative numbers. Figure~\ref{fig:cumulative}b shows the monthly number of active users and subreddits.

Our interest in this paper is not so much whether users survive as it is about the behavior of active users.  Thus, 
in general our analysis will look only at active users and subreddits in each month; those that are temporarily or permanently gone from Reddit are not included.  

\subsection{Identifying cohorts}

\begin{figure*}[!tb]
\centering
\subimage[width=0.48, scale=0.40]{./images/avr_posts_per_user_over_time_total.eps}
\subimage[width=0.48, scale=0.40]{./images/avr_posts_per_user_user_ref_total.eps}
\caption{In Figure (a), monthly average posts per active user over clock time. In Figure (b), monthly average posts per active users in the user-time referential, i.e., message creation time is measured relative to the user's first post.  Each tick in the x-axis is one year.  In both figures (and all later figures), we consider only active users during each month; users that are either temporarily or permanently away from Reddit are not included.}
\label{fig:overall_posts}
\end{figure*}

We define the ``\textbf{user's creation time}'' as the time of the first post made by that user.  Throughout this paper, we will use the notion of user cohorts, which will consist of users created in the same calendar year.

In many cases, we will look at the evolution of these cohorts. Since users can be created at any time during their cohort year, and our dataset ends in 2014, 
we are likely to have a variation on the data available for each user of up to one year, even though they are in the same cohort.  To deal with this, some of our cohorted analyses will consider only the overlapping time window for which we collect data for all users in a cohort.   This means that we are normally not going to include the 2014 cohort in our analyses.

Our data starts in October 2007, but Reddit existed before that. That means that, not only do we have incomplete data for the 2007 year (which compromises this cohort), but there might also be users and subreddits that show up in 2007 that were actually created in the previous years. Since we can not control for these, we will also omit 2007 cohort. We will, however, include 2007 in the overall analyses over time (the non cohorted ones) for two reasons: first, it does not have any direct impact on the results; second, we often compare the cohorted approach with a naive approach based on aggregation, and we would not expect a naive approach to do such filtering. 


\section{Average posts per user}
One common way to represent user activity in online communities is quantity: the number of posts people make over time. Approaches that consider the total number of posts per user in a particular dataset \cite{Gruhl2004} and that analyze the variation of the number of posts per user over time \cite{Guo2009} have been applied to online social networks.  In this section, we use this measure to address our first research question (\textbf{RQ1)}: how does the amount of users' activity change over time?

As we will see, both visualizing behavior relative to a user's creation time and using cohorts provide additional insight into posting activity in Reddit compared to a straightforward aggregate analysis based on calendar time.

\subsection{Calendar versus user-relative time}

Figure~\ref{fig:overall_posts}a shows that aggregate analysis, presenting the average number of posts per month by active users in that month.  Taken at face value, this 
suggests that over the first few years of Reddit, users became more active in posting, with per-user activity remaining more or less steady since mid-2011.

\looseness=-1

%% Sam 8: Reorganized the next 3 paragraphs, added information about throw away accounts
This average view hides several important aspects of users' activity dynamics. Previous work has looked into behavior relative to the user creation time. It has been shown that edge creation time in a social network relative to the user creation follows an exponential distribution \cite{Tomkins2008}. User lifetime, however, does not follow a exponential distribution and some types of user content generation follow a stretched exponential distribution \cite{Guo2009}. Throw-away accounts are one example of very short-lived users in Reddit \cite{Bergstrom2011}, for example. 

To address these characteristics, Figure~\ref{fig:overall_posts}b shows a view that emphasizes the trajectory over a user's lifespan rather than the community's.  To do this, we scale the x-axis not by clock time, as in Figure~\ref{fig:overall_posts}a, but by time since the user's first post: ``1'' on the x-axis refers to one year since the user's account first post, and so on. We call this the \textbf{time in the user referential}. One caution about interpreting graphs with time in the user referential is that the amount of data available rapidly decreases over time as users leave the community, meaning that values toward the right side of an individual data series are more subject to individual variation.  

%% Sam 11: Treating the misinterpretation as a hypothesis
%% Sam 12: The point of highlighting this as hypothesis is to point out at the end all the possible mistakes.
The evidence at this point supports the tempting hypothesis that the longer a user survives, the more posts they make (\textbf{H1}).  This hypothesis, however, is incorrect; we will present a more nuanced description of what is happening informed by cohort-based analyses.

\subsection{New cohorts do not catch up}

\begin{figure*}[!tb]
\centering
\subimage[width=0.48, scale=0.42]{./images/avr_posts_per_user_over_time_cohorts.eps}
\subimage[width=0.48, scale=0.42]{./images/avr_posts_per_user_cohorts.eps}
\caption{Figure (a) shows the average number of posts per active user over clock time and Figure (b) per active user in the user-time referential, both segmented by users' cohorts. The user cohort is defined by the year of the user's creation time.  For comparison, the black line in Figure (a) represents the overall average.}
\label{fig:avr_posts_per_user_over_time_cohorts}
\end{figure*}

%% Sam 12: Adding the first research question here because we start cohorting here, and that is the main strategy of the paper.
%% DC 14: Didn't work as-is since it's not parallel with the others, and was referred to in the more general way in the discussion/conclusion section, so added a general version earlier and changed the label for this one.
Figure~\ref{fig:overall_posts}b suggests that older users are more active than newer ones, raising the question of whether new users
eventually follow in older users' footsteps (\textbf{RQ1a}).  

Analyzing users' behavior by cohort is a reasonable way to address this question, and Figure~\ref{fig:avr_posts_per_user_over_time_cohorts}a shows a first attempt at this analysis.  We can already observe a significant cohort effect: users from later cohorts appear to level off at significantly lower posting averages than users from earlier ones.  It suggests that newer users likely will never be as active as older ones on average.  It also shows that surviving users are significantly more active than the overall average (the black line in the figure) would suggest.

However, Figure~\ref{fig:avr_posts_per_user_over_time_cohorts}a also has an awkward anomaly: a rapid rise in the average number of posts during each cohort's first calendar year, especially in December. Combining cohort segmentation with user-referential analysis, as in Figure~\ref{fig:avr_posts_per_user_over_time_cohorts}b, helps smooth out this anomaly and aligns cohorts with each other.  Doing this alignment makes clear that differences between earlier and later cohorts are apparent early on.

\subsection{Does tenure predict activity, or vice versa?}

\begin{figure*}[!tb]
\centering
\subimage[width=0.31, scale=0.29]{./images/avr_posts_per_user_for_surviving_year_for_2010.eps}{2010 cohort}
\subimage[width=0.31, scale=0.29]{./images/avr_posts_per_user_for_surviving_year_for_2011.eps}{2011 cohort}
\subimage[width=0.31, scale=0.29]{./images/avr_posts_per_user_for_surviving_year_for_2012.eps}{2012 cohort}
\caption{Each Figure corresponds to one cohort, from 2010 to 2012, left to right. The users for each cohort are further divided in groups based on how long they survived: users that survived up to 1 year are labeled 0, from 1 to 2 years are labeled 1, and so on.  For all cohorts, longer-tenured users started at higher activity levels than shorter-tenured ones.}
\label{fig:avr_posts_per_user_for_surviving_year}
\end{figure*}

%% Sam 11: Accommodating the hypothesis, adding the second ``correct'' hypothesis
\looseness=-1
These graphs still support our initial hypothesis \textbf{H1} 
%the tempting conclusion that users become more active the longer they exist in Reddit, 
and they do not explain the rapid increase in posting activity in the first few months.  An alternative hypothesis, inspired by the ``Wikipedians are Born, not Made'' paper \cite{Panciera2009}, is that individual users come in with different posting propensities, and the rise over time is not that individual users become more active but that low-activity users leave the system (\textbf{H2}).  To examine this, we further segment each cohort by the number of years they were active in the system, as defined by the difference between their first and last post times.
 
Figure~\ref{fig:avr_posts_per_user_for_surviving_year} shows this analysis for the 2010, 2011 and 2012 cohorts\footnote{We only show these figures for the sake of saving space, but the same trends are observed in the other cohorts.}.  Across all cohorts and yearly survival sub-cohorts, users who leave earlier come in with a lower initial posting rate.  Thus, the rise in average posts per active user is driven by the fact that users who have high posting averages throughout their lifespan are the ones who are more likely to survive.  As the less active users leave the system, the average per active user increases.  In other words, the correct interpretation of Figure~\ref{fig:overall_posts}b is not \textbf{H1}: longer-lived users don't post more as they age.  Instead, users who post more---right from the beginning---live longer, supporting (\textbf{H2}). 

Combining Figure~\ref{fig:avr_posts_per_user_for_surviving_year}'s insight that the main reason why these curves increase is because the low posting users are dying sooner with the earlier observation that the stable activity level is lower for newer cohorts suggests that low-activity users from later cohorts tend to survive longer than those from earlier cohorts.  That is, people joining later in the community's life are less likely to be either committed users or leave than those from earlier on: they are more likely to be ``casual'' users that stick around.

\vspace{7pt} 
\section{Comment length}

\begin{figure*}[!tb]
\centering
\subimage[width=0.48, scale=0.42]{./images/avr_comment_size_over_time_cohorts.eps}
\subimage[width=0.48, scale=0.42]{./images/avr_comment_size_cohorts.eps}
\subimage[width=0.31, scale=0.29]{./images/avr_comment_length_for_surviving_year_for_2010.eps}{2010 cohort}
\subimage[width=0.31, scale=0.29]{./images/avr_comment_length_for_surviving_year_for_2011.eps}{2011 cohort}
\subimage[width=0.31, scale=0.29]{./images/avr_comment_length_for_surviving_year_for_2012.eps}{2012 cohort}
\caption{Figure (a) shows the average comment length over clock time and Figure (b) from the user-referential time. Both figures show the cohorted trends.  The overall average length per comment decreases over time, although for any individual cohort, it increases after a sharp initial drop. Figures (c), (d) and (e), similar to Figure~\ref{fig:avr_posts_per_user_for_surviving_year}, show the monthly average comment length for active users in the cohorts of 2010, 2011 and 2012, segmented by the number of years that the user survived in the network.  Opposite the analysis for average posts, which showed that low-activity users were the first to leave Reddit, here, people who start out as longer commenters are \textit{more} likely to leave.}
\label{fig:comment_length}
\end{figure*}

Activity as measured by the average number of posts per user is one proxy for user effort.  Comment length can also be considered as a proxy for user effort in the network.  Users that type more put more of their time in the network, contribute with more content, and might create stronger ties with the community. Thus, we put forward the following question (\textbf{RQ2}): how does comment length change in the community over time, both overall and by cohort?

\subsection{Comment length drops over time}

%% Sam 11: Adding hypothesis tags. Later on to mention all the possible misleading hypothesis that we can come up with not considering cohorts and time
%% Sam 12: The point of highlighting this as hypothesis is to point out at the end all the possible mistakes.
%% DC 14: Fair enough, but will want to refer to them a little more throughout the text as needed.  In particular, H4 is maybe not a mistake, in the sense that a whole bunch of people coming in and acting differenly might be an instance of that behavior-defining-norms idea from previous work.
Figure~\ref{fig:comment_length}a shows the overall comment length in Reddit over time (the darker line) and the overall length per cohort. 
Based on the downwards tendency of the overall comment length in Figure~\ref{fig:comment_length}a, one might hypothesize that users' commitment to the network is decreasing over time (\textbf{H3}), or that there is some community-wide norm toward shorter commenting (\textbf{H4}). 

However, this might not be the best way to interpret this information. Figure~\ref{fig:comment_length}b shows the comment length per cohort in the user referential time. An important observation here is that younger users start from a lower baseline comment length than older ones. Considering the fact that Reddit has experienced exponential growth, the overall average for Figures \ref{fig:comment_length}a and \ref{fig:comment_length}b is heavily influenced by the ever-growing younger generations, who are more numerous than older survivors and who post shorter comments. 

\subsection{Simpson's Paradox: the length also rises}

Let us go back to Figure~\ref{fig:comment_length}a, which shows the overall average comment length on Reddit over time. We see a clear trend towards declining length of comments in the overall line (the black line that averages across all users). This could be a warning sign for Reddit community managers, assuming longer comments are associated with more involved users and healthier discussions. A data analyst looking at these numbers might think about ways to promote longer comments on Reddit. 

However, Figure~\ref{fig:comment_length}b shows that average comment length increases over time for every cohort. While later cohorts start at smaller comment length, after an initial drop, on average all cohorts write longer comments over time.  This is puzzling: when each of the cohorts exhibits a steady increase in their average comment length, how can the overall mean comment length decrease?  This anomaly is an instance of the Simpson's paradox \cite{simpson1951}, and occurs because we fail to properly condition on different cohorts when computing mean comment length. 

\begin{table}[!tb]
\centering
\tabcolsep=0.07cm
\singlespacing
\fontsize{9pt}{10.5pt}\selectfont
\begin{tabular}{|c|c|c|c|c|c|c|c|c|c|}
\cline{2-9}
\multicolumn{1}{c|}{} & \multicolumn{8}{c|}{Cohorts} \\ \hline
Year & 2007 & 2008 & 2009 & 2010 & 2011 & 2012 & 2013 & 2014 & Overall\\ \hline
2007 & 220 & - & - & - & - & - & - & - & 220 \\ \hline
2008 & 208 & 198 & - & - & - & - & - & - & 204 \\ \hline
2009 & 224 & 204 & 201 & - & - & - & - & - & 208 \\ \hline
2010 & 223 & 204 & 189 & 184 & - & - & - & - & 193 \\ \hline
2011 & 233 & 211 & 199 & 184 & 167 & - & - & - & 182 \\ \hline
2012 & 241 & 221 & 212 & 197 & 173 & 167 & - & - & 178 \\ \hline
2013 & 244 & 225 & 214 & 199 & 177 & 167 & 164 & - & 174 \\ \hline
2014 & 246 & 229 & 217 & 204 & 183 & 172 & 165 & 176 & 176 \\ \hline
\end{tabular}
\caption{Evolution of the average throughout the years for each cohort. Each column here is one cohort and each line is one year in time. Cohorts start generating data in their cohort year, therefore the upper diagonal is blank. On the right column we see the overall average for all users.}
\label{tab:simpson}
\end{table}

Table~\ref{tab:simpson} provides some clues to what might be going on. When we move down the rows, we observe an increasing tendency in each cohort column. It means that the average comment length increases for these users. However, when we move right through the columns, people in later cohorts tend to write less per comment. If we were to average each row, we would still get an overall increasing comment length per year, but that is not what we see in the overall column. What happens here is that the latter cohorts have many more users than earlier ones. Since their numbers increase year by year, we have a much larger contribution from them towards comments, compared to users of earlier cohorts. This uneven contribution leads to the paradox we observed in Figure~\ref{fig:comment_length}a. 

Without the decision to condition on cohorts, one would have gathered an entirely wrong conclusion. People are not writing less as they survive, contra (\textbf{H3}).  Rather, those who tend to write less are joining the community in much larger numbers.  Why later users write less is an open question we speculate about later in the discussion and future work section.
%% DC 14: Didn't feel so convinced by the first half, and the second half is redundant with my rewording above
% Knowing this, one may focus on better onboarding processes for newcomers, or try to learn why users in later cohorts tend to write smaller comments on average.  

\begin{figure*}[!tb]
\centering
\subimage[width=0.48, scale=0.42]{./images/comments_per_submissions_over_time_cohorts.eps}
\subimage[width=0.48, scale=0.42]{./images/comments_per_submissions_cohorts.eps}
\subimage[width=0.23, scale=0.23]{./images/comments_per_submissions_for_surviving_year_for_2008.eps}{2008 cohort}
\subimage[width=0.23, scale=0.23]{./images/comments_per_submissions_for_surviving_year_for_2009.eps}{2009 cohort}
\subimage[width=0.23, scale=0.23]{./images/comments_per_submissions_for_surviving_year_for_2010.eps}{2010 cohort}
\subimage[width=0.23, scale=0.23]{./images/comments_per_submissions_for_surviving_year_for_2011.eps}{2011 cohort}
\caption{Figure (a) shows the average comment per submission ratio over clock time for the cohorts and the overall average. Figure (b) shows the average comment per submission from the user-referential time for the cohorts. Figures (c), (d), (e) and (f), similarly to Figure~\ref{fig:avr_posts_per_user_for_surviving_year}, shows the 2008, 2009, 2010, and 2011 cohorts, segmented by the number of years a user in the cohort survived.  As with average posts per month, users who stay active longer appear to start their careers with a relatively higher comments per submission ratio than users who abandon Reddit sooner.  Unlike that analysis, however, the early 2008 cohort ends up below the later cohorts in Figure (b).}
\label{fig:comments_submissions}
\end{figure*}

\subsection{New users burn brighter}
As with the number of posts per user, we cannot say if the increase in the curves seen in \ref{fig:comment_length}b is due to lower-effort users dying first or because users are writing more as they live longer.  The sub-cohort analysis in \ref{fig:comment_length}c allows us to make two observations toward this question.  First, \textit{comment length does increase inside of each cohort}, no matter how long the user survives.  Second, as a general trend, \textit{users that make longer comments inside of each cohort die faster}. This is quite surprising, given that we would expect people to put less effort when they are more likely to stop using the network.
%% DC 14: Not sure why these conclusions are made italic when others are not.  

\section{Kinds of contributions}

In addition to questions of effort, the online community literature also often asks what sorts of activities users engage in, for instance, to categorize users into roles they play in the community \cite{Welser2011}. As with comment length, we propose the following research quetion (\textbf{RQ3}): how do users' activities change in the community over time, both overall and by cohort?

\subsection{Over time, responsiveness increases}
Consider the case of Usenet: people who never start threads and only respond play the role of answerer, while there are other roles that include fostering discussion \cite{Welser2007}.  These might naturally map onto people who primarily comment and who primarily submit in Reddit, respectively.  Submissions can be considered new content that an author generates, while comments can be considered as contributions toward existing content from another author.

Since the total number of comments always surpasses the number of submissions, we compute a user's ratio of comments per submission as a rough measure of the kinds of contributions they make.  Figure~\ref{fig:comments_submissions}a shows the overall and cohorted evolution of comments per submission from 2008 to 2013.  Users who most prefer commenting to submitting come from 2009 to 2011, while over time the average ratio of comments to submissions increases both overall and per-cohort for active users.

Again, we analyze our data from the user-time referential, as seen in Figure~\ref{fig:comments_submissions}b. It shows a clear pattern for users in earlier cohorts to have a lower comment per submission ratio than users in later cohorts, given that they both survived the same amount of time.  Surviving users from later cohorts also exhibit a more rapid increase in comments per submission than those from earlier cohorts.  In particular, the 2008 and 2009 cohorts increase much more slowly over time than those from 2010 onwards; later cohorts are more similar (although the 2012 and 2013 cohorts may level off lower than 2011 based on the limited data we have). 

%% DC 14: To be parallel to other sections, are there smaller hypotheses running around here, about roles changing over time: that people start as submitters and become commenters, or that comments and submissions might be complements or supplements?
\vfill\eject
\subsection{Comment early, comment often}

Figures~\ref{fig:comments_submissions}c-f shows the cohorts from 2008 to 2011 segmented by surviving year.  Three interesting observations arise from these data.  First, we see that just as in the analysis of average posts per user, the users who survive the longest in each cohort are the ones who hit the ground running.  They start out with a high comment-to-submission ratio relative to users in their cohort who abandon Reddit more quickly.  This suggests that both the count of posts and the propensity to comment might be a useful early predictor of user survival.

Second, and unlike the case for average post length, surviving users' behavior changes over time.  For post length, Figure~\ref{fig:avr_posts_per_user_for_surviving_year} shows that even the most active users come in at a certain activity level and stay there, perhaps even slowly declining over time.  Here, Figures~\ref{fig:comments_submissions}c-f show that the ratio of comments to submissions increases over time.  Combined with the observation that overall activity stays steady, this suggests that the ratio is changing because people \textit{substitute} making their own submissions for commenting on others' posts.

Finally, this increase is most pronounced in the earlier cohorts of 2008 and 2009, with ratios more than doubling over their first year, much more than for later cohorts.
%% Sam 11: I don't quite remember why we wrote this, is it right? I can't really see how this is true from the figures.
%Still, the ratio for these earlier cohorts never rises to the level it does for surviving users from later cohorts. 


